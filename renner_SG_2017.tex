\documentclass[11pt,a4paper]{article}
%\pagestyle{headings}
%\linespread{1.6}

% Set up 20 mm margins on each side
\hoffset -5.4mm
\voffset -5.4mm

\oddsidemargin 0.0in
\evensidemargin 0.0in

\textwidth 17.0cm
\textheight 232mm

\topmargin 0.0in
\headheight 4mm
\headsep 9mm
\footskip 12mm

%\usepackage{hyperref,color,graphicx,braket,mathrsfs,simplewick,slashed,amsmath,amssymb,feynmf,ifpdf,pdflscape,wrapfig,fancyhdr,lastpage,etoolbox}
\usepackage{hyperref,xcolor,graphicx,ifpdf,wrapfig,fancyhdr,lastpage,etoolbox,tocloft,multirow,array,pdflscape,enumitem,eurosym,amssymb,titlesec}
\usepackage[square,numbers]{natbib}
%\usepackage[backend=bibtex,style=authoryear]{biblatex}
%\addbibresource{/home/josh/Dropbox/bibliography/xenon_fulbright.bib}

\hypersetup{colorlinks,%
            citecolor=black,%
            linkcolor=black,%
            filecolor=black,%
            urlcolor=black}

\bibliographystyle{apsrev4-1}

% Set up the header and footer.
\makeatletter
\patchcmd{\@fancyhead}{\rlap}{\color{black}\rlap}{}{}
\patchcmd{\@fancyfoot}{\rlap}{\color{black}\rlap}{}{}
\makeatother

% Set up the title spacing
\titlespacing\section{0pt}{4pt plus 4pt minus 2pt}{0pt plus 2pt minus 2pt}
\titlespacing\subsection{0pt}{4pt plus 4pt minus 2pt}{0pt plus 2pt minus 2pt}
\titlespacing\subsubsection{0pt}{4pt plus 4pt minus 2pt}{0pt plus 2pt minus 2pt}

\pagestyle{fancy}
\fancyhf{}
\renewcommand{\headrulewidth}{0pt}
\lhead{Renner}
\chead{Part B1}
\rhead{DNNEXT}
\cfoot{\thepage}

% To keep the numbering of the table of contents consistent.
\tocloftpagestyle{fancy}

% Table column formatting: from 
% http://tex.stackexchange.com/questions/12703/how-to-create-fixed-width-table-columns-with-text-raggedright-centered-raggedlef
\newcolumntype{L}[1]{>{\raggedright\let\newline\\\arraybackslash\hspace{0pt}}m{#1}}
\newcolumntype{C}[1]{>{\centering\let\newline\\\arraybackslash\hspace{0pt}}m{#1}}
\newcolumntype{R}[1]{>{\raggedleft\let\newline\\\arraybackslash\hspace{0pt}}m{#1}}

\begin{document}
% BB
\newcommand{\bb}{\ensuremath{\beta\beta}}
% BB0NU
\newcommand{\bbonu}{\ensuremath{\beta\beta0\nu}}
% BB2NU
\newcommand{\bbtnu}{\ensuremath{\beta\beta2\nu}}
% NME
\newcommand{\Monu}{\ensuremath{\Big|M_{0\nu}\Big|}}
\newcommand{\Mtnu}{\ensuremath{\Big|M_{2\nu}\Big|}}
% PHASE-SPACE FACTOR
\newcommand{\Gonu}{\ensuremath{G^{0\nu}(\Qbb, Z)}}
\newcommand{\Gtnu}{\ensuremath{G^{2\nu}(\Qbb, Z)}}

% mbb
\newcommand{\mbb}{\ensuremath{m_{\beta\beta}}}
\newcommand{\kgy}{\ensuremath{\rm kg \cdot y}}
\newcommand{\ckky}{\ensuremath{\rm counts/(keV \cdot kg \cdot yr)}}
\newcommand{\mbba}{\ensuremath{m_{\beta\beta}^a}}
\newcommand{\mbbb}{\ensuremath{m_{\beta\beta}^b}}
\newcommand{\mbbt}{\ensuremath{m_{\beta\beta}^t}}
\newcommand{\nbb}{\ensuremath{N_{\beta\beta^{0\nu}}}}

% Qbb
\newcommand{\Qbb}{\ensuremath{Q_{\beta\beta}}}

% Tonu
\newcommand{\Tonu}{\ensuremath{T_{1/2}^{0\nu}}}

% Tonu
\newcommand{\Ttnu}{\ensuremath{T_{1/2}^{2\nu}}}

% Xe-136
\newcommand{\Xe}{\ensuremath{^{136}}Xe}
\newcommand{\COT}{\ensuremath{CO_2}}
\newcommand{\CHF}{\ensuremath{CH_4}}
\newcommand{\CFF}{\ensuremath{CF_4}}

% 2S
\newcommand{\TwoS}{\ensuremath{^{2}S_{1/2}}}

\newcommand{\TwoP}{\ensuremath{^{2}P_{1/2}}}

\newcommand{\TwoD}{\ensuremath{^{2}D_{3/2}}}


% Xe-136
\newcommand{\CS}{\ensuremath{^{137}}Cs}

% Xe-136
\newcommand{\NA}{\ensuremath{^{22}}Na}


% Bi-214
\newcommand{\Bi}{\ensuremath{^{214}}Bi}

% Tl-208
\newcommand{\Tl}{\ensuremath{^{208}}Tl}

% Pb-208
\newcommand{\Pb}{\ensuremath{^{208}}Pb}
% Pb-208
\newcommand{\PBD}{\ensuremath{^{210}}Pb}

% Po-214
\newcommand{\Po}{\ensuremath{^{214}}Po}
\newcommand{\Kr}{\ensuremath{^{83}}Kr}

% bru
\newcommand{\bru}{cts/(keV$\cdot$kg$\cdot$y)}
\newcommand{\dten}{10 mm/$\sqrt{\rm m}$}
\newcommand{\dtwo}{2 mm/$\sqrt{\rm m}$}
\newcommand{\BAPP}{\ensuremath{Ba^{++}}}
\newcommand{\BAP}{\ensuremath{Ba^{+}}}

\newcommand{\HPXE}{\sc{HPXe}\rm}
\newcommand{\BATA}{\sc{BaTa}\rm}

% Saltos de carro en tablas
\newcommand{\minitab}[2][l]{\begin{tabular}{#1}#2\end{tabular}}

\newcommand{\thedraft}{0.1.1}% version for referees

\newcommand{\MO}{\ensuremath{{}^{100}{\rm Mo}}}
\newcommand{\SE}{\ensuremath{{}^{82}{\rm Se}}}
\newcommand{\ZR}{\ensuremath{{}^{96}{\rm Zr}}}
\newcommand{\KR}{\ensuremath{{}^{82}{\rm Kr}}}
\newcommand{\ND}{\ensuremath{{}^{150}{\rm Nd}}}
\newcommand{\XE}{\ensuremath{{}^{136}\rm Xe}}
\newcommand{\GE}{\ensuremath{{}^{76}\rm Ge}}
\newcommand{\GES}{\ensuremath{{}^{68}\rm Ge}}
\newcommand{\TE}{\ensuremath{{}^{128}\rm Te}}
\newcommand{\TEX}{\ensuremath{{}^{130}\rm Te}}
\newcommand{\TL}{\ensuremath{{}^{208}\rm{Tl}}}
\newcommand{\CA}{\ensuremath{{}^{48}\rm Ca}}
\newcommand{\CO}{\ensuremath{{}^{60}\rm Co}}
\newcommand{\PO}{\ensuremath{{}^{214\rm Po}}}
\newcommand{\U}{\ensuremath{{}^{235}\rm U}}
\newcommand{\CT}{\ensuremath{{}^{10}\rm C}}
\newcommand{\BE}{\ensuremath{{}^{11}\rm Be}}
\newcommand{\BO}{\ensuremath{{}^{8}\rm Be}}
\newcommand{\UDTO}{\ensuremath{{}^{238}\rm U}}
\newcommand{\CD}{\ensuremath{^{116}{\rm Cd}}}
\newcommand{\THO}{\ensuremath{{}^{232}{\rm Th}}}
\newcommand{\BI}{\ensuremath{{}^{214}}Bi}


\begin{center}
\large
\textbf{ERC Starting Grant 2017}\\
Research proposal [Part B1]\\[2.0\baselineskip]
\Large
\textbf{Deep Learning in Detector Physics Analyses}\\[1.0\baselineskip]
\LARGE
\textbf{DNNEXT}\\[1.0\baselineskip]
\end{center}
% \vspace{5.5 in}

\noindent PI: Joshua Edward Renner\\
\noindent Host Institution: Universidad de Valencia\\
\noindent Proposal Duration: 60 months\\

The development of deep learning has been realized over the past 10 years with the rise of computing power, which has now permitted the training of neural networks with many layers
of neurons in series.  Such networks have proven capable of high-level abstraction, that is, neurons in deeper layers have been shown to indicate the presence of complex features in a dataset, 
such as the presence of macroscopic objects in images.  It is of great interest to study how such techniques can be applied to physics, and in particular particle detection and tracking, a field 
in which problems exist similar to those to which deep learning has been initially applied.  We propose an in-depth study of the use of deep neural networks in several applications of particle
physics.  We discuss the results that have already been obtained which indicate the advantages of deep neural networks over classical analysis methods and highlight their merit for further study.
\newpage
\normalsize

\newpage
%\setcounter{page}{1}
{\textbf{Section A: Extended Synopsis of the scientific proposal}}

\newpage
\noindent{\textbf{Section B: Curriculum Vitae: Joshua Edward Renner (Postdoctoral Researcher)}}\\[1.0\baselineskip]
\textbf{Renner, Joshua Edward}\\
\textbf{Date of birth:} 30/08/1987\\
\textbf{Nationality:} United States of America\\
\textbf{Email:} jrenner@ific.uv.es\\

%{\noindent\textbf{Research interests}}
%
%\indent\hspace{0.2 cm}\textbullet\,\,Design, analysis, and simulation of particle detectors\\
%\indent\hspace{0.2 cm}\textbullet\,\,Neutrinoless double-beta decay\\
%\indent\hspace{0.2 cm}\textbullet\,\,Deep learning\\

{\noindent\textbf{EDUCATION}}\\

\begin{tabular}{ll}
May 2014 & Ph.D. in Physics\\
& Department of Physics, University of California, Berkeley, CA (U.S.A.)\\
& Advisor: Prof.\ James Siegrist\\

May 2011 & Master of Arts in Physics\\
& Department of Physics, University of California, Berkeley, CA (U.S.A.)\\

May 2009 & Bachelor of Science in Physics\\
& Department of Physics, Georgia Institute of Technology, Atlanta, GA (U.S.A.)\\
\end{tabular}\\

{\noindent\textbf{CURRENT POSITION}}\\

\begin{tabular}{ll}
	July 2016 - & \emph{Investigador Doctor Senior} (Postdoctoral Researcher)\\
	& Instituto de F\'{i}sica Corpuscular (IFIC), University of Valencia, Spain\\
\end{tabular}\\

{\noindent\textbf{PREVIOUS POSITIONS}}\\

\begin{tabular}{ll}
	Sep 2015 - Jun 2016 & \emph{Fulbright Postdoctoral Researcher}\\
	& Instituto de F\'{i}sica Corpuscular (IFIC), University of Valencia, Spain\\
	& and Fulbright Espa\~{n}a\\

	Oct 2014 - Jul 2015 & \emph{Investigador Doctor Junior} (Postdoctoral Research)\\
	& Instituto de F\'{i}sica Corpuscular (IFIC), University of Valencia, Spain\\
	
	May 2009 - May 2014 & \emph{Graduate Student Researcher}\\
	& Lawrence Berkeley National Laboratory (LBNL) and\\
	& University of California, Berkeley, CA (U.S.A.)\\
	& - supported by DOE NNSA SSGF Fellowship Sep 2009 - Aug 2014\\
	& - at Lawrence Livermore National Laboratory May 2010 - Aug 2010\\
	
	May 2007 - Aug 2007, & \emph{Student Employee}\\
	Jan 2008 - Aug 2008 & Georgia Tech Research Institute (GTRI), Atlanta, GA\\
	& Radar Warning Receiver Division\\
\end{tabular}\\

{\noindent\textbf{FELLOWSHIPS}}\\

\begin{tabular}{ll}
	Sep 2015 - Jun 2016 & \emph{Fulbright Junior Research Award}\\
	& Fulbright Espa\~{n}a, Madrid, Spain\\
	& - 9-month research award to continue work on NEXT\\
	
	Sep 2009 - Aug 2014 & \emph{DOE NNSA SSGF Fellowship}\\
	& U.S. Department of Energy (DOE)\\
	& National Nuclear Security Administration (NNSA)\\
	& Stewardship Science Graduate Fellowship (SSGF)\\
	& - 4-year research award covering tuition, fees, and monthly stipend\\
	& - See: http://www.krellinst.org/ssgf\\
	
\end{tabular}\\

\newpage
{\noindent\textbf{TEACHING ACTIVITIES}}\\

\begin{tabular}{ll}
	
	Jul 2016 & \emph{Tutor: IFIC Summer School}\\
	& Instituto de F\'{i}sica Corpuscular (IFIC), University of Valencia, Spain\\
	& - Mentored two undergraduate students during a 2-week summer school\\
	
	Mar 2016 - Apr 2016 & \emph{Tutor: Experimental Project, Masters of Theoretical Physics}\\
	& Instituto de F\'{i}sica Corpuscular (IFIC), University of Valencia, Spain\\
	& - Mentored one Masters student during a 3-week experimental project\\
		
	Aug 2009 - Dec 2009 & \emph{Graduate Student Instructor}\\
	& Dept. of Physics, University of California, Berkeley, CA (U.S.A.)\\
	& - Course: Physics 111, Basic Semiconductor Circuits Laboratory\\
\end{tabular}\\

{\noindent\textbf{WORKSHOPS AND SPECIALIZED SCHOOLS ATTENDED}}\\

\begin{tabular}{ll}
	Sep 4-8, 2013 & \emph{TAUP 2013 Summer School on Astroparticle and Underground Science}\\
	& Asilomar Conference Grounds, Asilomar, CA (U.S.A.)\\
	& - Lecture series\\
	
	Feb 13-24, 2012 & \emph{Excellence in Detectors and Instrumentation Technologies (EDIT) Symposium}\\
	& Fermi National Accelerator Laboratory, Batavia, IL (U.S.A.)\\
	& - Series of lectures and laboratory courses
\end{tabular}\\

\noindent \textbf{PUBLICATIONS AS LEAD AUTHOR AND/OR PRINCIPAL CONTRIBUTOR}\\

\noindent J. Renner, A. Cervera, J. A. Hernando, A. Imzaylov, F. Monrabal, J. Mu\~noz, D. Nygren, and J.J. Gomez-Cadenas.  Improved background rejection in neutrinoless double beta decay experiments using a magnetic field in a high pressure xenon TPC.  JINST 10 (2015) P12020. (arXiv:1509.01821)\\

\noindent J. Renner \emph{et al.} (NEXT Collaboration).  Ionization and scintillation of nuclear recoils in gaseous xenon.  Nucl. Instr. Meth. A 793, 62 (2015).  (arXiv:1409.2853)\\

\noindent V.\ \'{A}lvarez \emph{et al.} (NEXT Collaboration). Near-intrinsic energy resolution for 30 to 662 keV gamma rays in a high pressure xenon electroluminescent TPC. Nucl. Instr. Meth. A 708, 
101 (2013).\\(arXiv:1211.4474)\\
% available online January 18, 2013

\noindent \textbf{SELECTED CONFERENCE PRESENTATIONS}\\

\noindent J. Renner, V. M. Gehman, A. Goldschmidt, D. Nygren, and C.A.B. Oliveira, for the NEXT Collaboration. 
Characterization of Nuclear Recoils in High Pressure Xenon Gas: Towards a Simultaneous Search for WIMP Dark Matter and Neutrinoless 
Double Beta Decay.  TAUP 2013 (presentation by J. Renner on Sept. 11, 2013). Phys. Procedia 61, 766 (2015).\\

\noindent Nuclear Recoils and Recombination in High Pressure Xenon Gas.  Advances in Neutrino Technologies (ANT) 2013 
(presentation by J. Renner on May 11, 2013).\\

\noindent A. Goldschmidt, T. Miller, D. Nygren, J. Renner, D. Shuman, H. Spieler, and J. White.
High-pressure xenon gas TPC for neutrino-less double-beta decay in 136Xe: Progress toward the goal of 1\% FWHM energy resolution.” 
2011 IEEE Nuclear Science Symposium (presentation by J. Renner on Oct. 26, 2011).\\

%{\noindent\textbf{MAJOR COLLABORATIONS}}\\
%
%\indent\hspace{0.2 cm}\textbullet\,\,Assisted in construction and operation of LBNL prototype detector\\
%\indent\hspace{0.2 cm}\,\,\,\,\,\,\,- Primary author of automated data readout and analysis code\\
%\indent\hspace{0.2 cm}\,\,\,\,\,\,\,- Worked on detector electronics and gas system, and PMT calibration\\
%\indent\hspace{0.2 cm}\textbullet\,\,Assisted in mentoring undergraduate students working in the LBNL group\\
%\indent\hspace{0.2 cm}\textbullet\,\,Played a key role in demonstrating 1\% FWHM energy resolution at 662 keV for electrons\\
%\indent\hspace{0.2 cm}\,\,\,\,\,\,\,\,in high pressure xenon gas\\
%\indent\hspace{0.2 cm}\textbullet\,\,Assisted in development of detector Monte Carlo and its execution in supercomputing\\
%\indent\hspace{0.2 cm}\,\,\,\,\,\,\,\,environment\\
%\indent\hspace{0.2 cm}\textbullet\,\,Measured ionization and scintillation produced by nuclear recoils in high pressure xenon\\
%\indent\hspace{0.2 cm}\,\,\,\,\,\,\,\,gas\\
%\indent\hspace{0.2 cm}\textbullet\,\,Developed algorithms to improve background rejection based on event topology\\

\newpage
{\textbf{\emph{Appendix: All on-going and submitted grants and funding of the PI (Funding ID)}}}\\

\noindent\textbf{On-going grants}\\

\noindent\textbf{Grant applications}\\

\newpage
{\textbf{Section C: Early achievements track-record}}

\newpage
\chead{Part B2}
\begin{center}
	\large
	\textbf{ERC Starting Grant 2017}\\
	Research proposal [Part B1]\\[2.0\baselineskip]
\end{center}

\noindent\textbf{Part B2: \underline{The scientific proposal}}\\

{\noindent\textbf{Section A: State-of-the-art and objectives}}\\
In recent years, increasingly sensitive null results (cite) in searches for neutrinoless double beta ($0\nu\beta\beta$) decay have made it clear that an ultra-low background experiment 
employing on the order of tonnes of active mass will be necessary to have a meaningful chance of discovery.  Such an experiment would require excellent energy resolution as well as the use
additional mechanisms for background rejection.  Gaseous xenon enriched in the isotope $^{136}$Xe has shown strong potential for being the 
isotope of choice for a discovery experiment due to its relatively low cost, ability to act as both source and detector, outstanding energy resolution, and the long ionization tracks produced by
electrons at energies of order $Q_{\beta\beta}$, which can be analyzed to provide increased background rejection.  Currently ...\\

The conventional analysis may not be enough; also want to show how DNNs can be used in physics

Discuss NEXT, DNNs, and physics applications of DNNs (what has already been done?).  Objectives:
\begin{enumerate}
	\item Understand the workings and limitations of DNNs (show DIGITS plots of weights)
	\item Produce a DNN-based analysis framework for NEXT
	\item Explore further applications of DNNs in particle physics (angular distribution in NEXT?)
\end{enumerate}
It needs to be made clear that NEXT is critically dependent on such an analysis, and it may be essential for a future $0\nu\beta\beta$ decay experiment.

{\noindent\textbf{Section B: Methodology}}\\
Discuss use of Python (Keras, TF) and previous use of Caffe/DIGITS.  Discuss use of GPUs.  Develop plan for analysis and procedure for understanding DNNs.  Show what has already been done.\\

{\noindent\textbf{Section C: Resources (including project costs)}}\\

\end{document}